\documentclass[conference]{IEEEtran}
\IEEEoverridecommandlockouts
% The preceding line is only needed to identify funding in the first footnote. If that is unneeded, please comment it out.
\usepackage{amsmath,amssymb,amsfonts}
\usepackage{algorithmic}
\usepackage[backend=biber]{biblatex}
\usepackage{graphicx}
\usepackage{textcomp}
\usepackage{xcolor}
\addbibresource{bibliography.bib}
\def\BibTeX{{\rm B\kern-.05em{\sc i\kern-.025em b}\kern-.08em
    T\kern-.1667em\lower.7ex\hbox{E}\kern-.125emX}}

\begin{document}

\title{Assessment of a MACsec-based security system for use in critical Infrastructure Communication}

\author{\IEEEauthorblockN{Lukas F{\"u}reder}
    \IEEEauthorblockA{\textit{Technical University of Applied Sciences Regensburg (OTH)} \\
        \textit{Laboratory for Safe and Secure Systems (LaS³)}\\
        Regensburg, Germany \\
        lukas.fuereder@oth-regensburg.de}
}

\maketitle

%%%%%%%%%%%%%%%%%%%%%%%%%%%%%%%%%%%%%%%%%%%%%%%%%%%%%%%%%%%%%%          Abstract           %%%%%%%%%%%%%%%%%%%%%%%%%%%%%%%%%%%%%%%%%%%%%%%%%%%%%%%%%%%
\begin{abstract}
    Lorem ipsum dolor sit amet, consetetur sadipscing elitr, sed diam nonumy eirmod tempor invidunt ut labore et dolore magna aliquyam erat, sed diam 
    voluptua. At vero eos et accusam et justo duo dolores et ea rebum. Stet clita kasd gubergren, no sea takimata sanctus est Lorem ipsum dolor sit 
    amet. Lorem ipsum dolor sit amet, consetetur sadipscing elitr, sed diam nonumy eirmod tempor invidunt ut labore et dolore magna aliquyam erat, sed 
    diam voluptua. At vero eos et accusam et justo duo dolores et ea rebum. Stet clita kasd gubergren, no sea takimata sanctus est Lorem ipsum dolor 
    sit amet. Lorem ipsum dolor sit amet, consetetur sadipscing elitr, sed diam nonumy eirmod tempor invidunt ut labore et dolore magna aliquyam erat, 
    sed diam voluptua. At vero eos et accusam et justo duo dolores et ea rebum. Stet clita kasd gubergren, no sea takimata sanctus est Lorem ipsum dolor 
    sit amet.
\end{abstract}
%%%%%%%%%%%%%%%%%%%%%%%%%%%%%%%%%%%%%%%%%%%%%%%%%%%%%%%%%%%%%%%%%%%%%%%%%%%%%%%%%%%%%%%%%%%%%%%%%%%%%%%%%%%%%%%%%%%%%%%%%%%%%%%%%%%%%%%%%%%%%%%%%%%%%%

\begin{IEEEkeywords}
    MACsec, IEC61850, IEC62351, GOOSE, Secure Communication
\end{IEEEkeywords}

%%%%%%%%%%%%%%%%%%%%%%%%%%%%%%%%%%%%%%%%%%%%%%%%%%%%%%%%%%%%%%      Start of the Text      %%%%%%%%%%%%%%%%%%%%%%%%%%%%%%%%%%%%%%%%%%%%%%%%%%%%%%%%%%%
\section{Introduction}
\label{chapter:introduction}
\noindent Companies that are classified as critical infrastructure as for example water supply facilities, power plants and their corresponding 
distribution systems, can constitute a vulnerability which may be exploited to disrupt the supply of basic resources to entire countries. For this reason, 
laws such as the Network and Information Security Act (NIS-2) \cite{NIS-2:2022} of the European Union or the IT Act 2.0 \cite{IT-Gesetz_2:2021} of the 
German Federal Office for Information Security (BSI) demand a unified level of cybersecurity for these entities. In these regulations, the councils 
prescribe that the companies will be required to implement security features to detect and prevent intrusions, as well as remove faults caused through 
intrusion attempts during system runtime. \cite[§11 (1d)]{IT-Gesetz_2:2021} Additionally the extension of this paragraph dictates, that these companies 
are obliged to provide proof of compliance with the safety requirements in a two year period. \cite[§11 (1e)]{IT-Gesetz_2:2021} This decision is intended 
to ensure the future working of the security systems with respect to adapting changes of the latest technologies. 

\smallskip
This paper evaluates the currently established implementation of protection systems securing communication in Substation Automation Systems (SASs) and 
thereby provides a brief overview of the communication standard used in these facilities. Following this we propose a Media Access Control Security 
(MACsec) based security system with the security goals set for these applications. The further course of the paper is structured as follows: Chapter 
\ref{chapter:relatedWork} displays relevant information presented by related works assessing the current state of technology in this topic. Chapter 
\ref{chapter:fundamentalsIEC} provides a general overview of the IEC 61850 communication standard and the associated IEC 62351 safety standard with 
special focus placed on the different message types and their respective protection. Chapter \ref{chapter:implementation} explains the test setup used 
to measure the efficiency of the MACsec-based security system. Lastly the data gathered from this is then evaluated in chapter \ref{chapter:evaluation}. 

%%%%%%%%%%%%%%%%%%%%%%%%%%%%%%%%%%%%%%%%%%%%%%%%%%%%%%%%%%%%%%        Fundamentals         %%%%%%%%%%%%%%%%%%%%%%%%%%%%%%%%%%%%%%%%%%%%%%%%%%%%%%%%%%%
\section{Background}


\subsection{Overview of the IEC 61850 \& IEC 62351 Standard}
\label{chapter:fundamentalsIEC}
Among other stadards used for communication is SASs, the facilities utilize the IEC 61850 standard, which is published and maintained by the International 
Electrotechnical Comission (IEC). This standard is used to transmit diagnostical information, measurement information or control signals between 
Supervisory Control and Data Acquisition (SCADA) entities and the associated substation components. The major advantage here consists of the object-oriented 
data structure specified in this standard, which enables the integration of various components developted by different vendors \cite[p. 5643]{Review_IEC62351:2019}. 

\begin{verbatim}
-- HIER NOCH WEITER MIT IEC61850 & IEC62351
\end{verbatim}

\subsection{Fundamentals of the MACsec Security System}

\begin{verbatim}
-- HIER NOCH WEITER MIT MACsec
\end{verbatim}


%%%%%%%%%%%%%%%%%%%%%%%%%%%%%%%%%%%%%%%%%%%%%%%%%%%%%%%%%%%%%%        Related Work         %%%%%%%%%%%%%%%%%%%%%%%%%%%%%%%%%%%%%%%%%%%%%%%%%%%%%%%%%%%
\section{Related Works}
\label{chapter:relatedWork}
\noindent To assess the operating principal of a MACsec-based security system in IEC 61850 compliant communication it is necessary to understand both 
the working method of the communication inside a substation as well as the corresponding functionality of the MACsec security standard. The following 
related works display these important aspects and are therefore relevant for the implementation of an experimental set up for MACsec secured industrial 
communication.

\smallskip
Mackiewicz \cite{IEC61850_Overview:2006} describes the overall usage of the IEC 61850 protocol by displaying key features as well as the general aspects 
of IEC 61850 compliant communication. Since this standard represents a core part of the communication inside of power grid systems, it is vital to 
understand the corresponding aspects such as communication paths, model structures or data addressing in order to design a representative test environment. 

\smallskip
Hussain  \textit{et al.} \cite{Review_IEC62351:2019} published a paper assessing the IEC 62351 standard and its security mechanisms towards IEC 61850 
compliant messaging. The publication initially describes the basic values and security goals of the safety standard and, building on this, which attacks 
can potentially be carried out on IEC 1850 messages to manipulate the internal workings of a SAS. At this point the paper primarily focuses on the 
Ethernet-based message types Generic Object Oriented Substation Event (GOOSE) and Sampled Values (SV) and the associated decision not to encrypt them due 
to strict time delivery requirements.  

\smallskip
Moreira  \textit{et al.} \cite{Cybersecurity_Substation:2016} evaluate various approaches to introduce cyber security in SASs. Initially, a brief outline 
of the communication structures in substations is presented. Building on this, various established security approaches are explained and evaluated based 
on the protection objectives of the IEC 62351 standard. The authors also point out possible implementation problems, such as incompatibilities between 
the security systems and the communication protocols or the handling of redundant packets inside ring-topology networks. In the further course of the 
paper, they present the idea of MACsec based communication security in SASs and the and the associated advantages and challenges that arise with it. 

\smallskip
Lackorzynski \textit{et al.} \cite{MACsecIndustrialOptimization:2020} proposed modifications of the IEEE 802.1AE standard to improve MACsec for usage in 
industrial applications. In particular, the fragmentation of Ethernet frames was considered. This procedure is necessary, if messages exceed the Maximum 
Transmission Unit (MTU) and are thus possibly discarded by the recipient of the message. The presented implementation ensures this parameter and spits 
messages into multiple frames, if it is exceeded. Additionally the authors discuss the usage of different cipher suits instead of the \texttt{AES-GCM 128/256} 
specified in the MACsec standard. The evaluation of their study shows that the ChaCha20-Poly1305 cipher is a promising alternative for industrial 
applications.

\smallskip 
Building on the findings of Moreira \cite{Cybersecurity_Substation:2016} and Lackorzynski \cite{MACsecIndustrialOptimization:2020} we formulate the 
evaluation of MACsec carried out for use in substations and other power systems based on the IEC 61850 standard. Along with this, we discuss the 
advantages and disadvantages of MACsec in comparison with the security goals of the IEC 62351 standard based on our findings. 

%%%%%%%%%%%%%%%%%%%%%%%%%%%%%%%%%%%%%%%%%%%%%%%%%%%%%%%%%%%%%%       Implementation        %%%%%%%%%%%%%%%%%%%%%%%%%%%%%%%%%%%%%%%%%%%%%%%%%%%%%%%%%%%
\section{Implementation}
\label{chapter:implementation}


%%%%%%%%%%%%%%%%%%%%%%%%%%%%%%%%%%%%%%%%%%%%%%%%%%%%%%%%%%%%%%         Evaluation          %%%%%%%%%%%%%%%%%%%%%%%%%%%%%%%%%%%%%%%%%%%%%%%%%%%%%%%%%%%
\section{Evaluation}
\label{chapter:evaluation}


%%%%%%%%%%%%%%%%%%%%%%%%%%%%%%%%%%%%%%%%%%%%%%%%%%%%%%%%%%%%%%         Conclusion          %%%%%%%%%%%%%%%%%%%%%%%%%%%%%%%%%%%%%%%%%%%%%%%%%%%%%%%%%%%
\section{Conclusion}
\label{chapter:conclusion}


%%%%%%%%%%%%%%%%%%%%%%%%%%%%%%%%%%%%%%%%%%%%%%%%%%%%%%%%%%%%%%     Begin of References      %%%%%%%%%%%%%%%%%%%%%%%%%%%%%%%%%%%%%%%%%%%%%%%%%%%%%%%%%%
\printbibliography
%%%%%%%%%%%%%%%%%%%%%%%%%%%%%%%%%%%%%%%%%%%%%%%%%%%%%%%%%%%%%%%%%%%%%%%%%%%%%%%%%%%%%%%%%%%%%%%%%%%%%%%%%%%%%%%%%%%%%%%%%%%%%%%%%%%%%%%%%%%%%%%%%%%%%%

\end{document}
